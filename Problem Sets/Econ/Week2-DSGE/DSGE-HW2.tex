\documentclass[12pt]{article}
%\topmargin -0.5in \textheight 600pt\textwidth 465pt \oddsidemargin
%10pt\evensidemargin 10pt
\topmargin -0.8in \textheight 650pt\textwidth 490pt \oddsidemargin
-8pt\evensidemargin 0pt

\usepackage{graphicx}
\usepackage{epstopdf}
\usepackage{amsthm}
\usepackage{amsmath}
\usepackage{amssymb}
\usepackage{amstext}
\usepackage{amsfonts}
\usepackage{enumerate}
\usepackage[authoryear,nonamebreak,bibstyle]{natbib}
%\usepackage{geometry, pdflscape}
\usepackage{footnote}
\makesavenoteenv{tabular}
\makesavenoteenv{table}
%\usepackage{rotating}
%\usepackage[onehalfspacing]{setspace}
\usepackage[shortlabels]{enumitem}
\usepackage{multirow}
\usepackage[title, toc, page]{appendix}
\usepackage{xcolor}
\usepackage{setspace}
\usepackage{cancel}
\usepackage{soul}

%\usepackage[toc,page]{appendix}

\usepackage{sectsty}
\subsubsectionfont{\normalfont\itshape}

\begin{document}
\baselineskip=.22in\parindent=30pt

\newtheorem{tm}{Theorem}
\newtheorem{dfn}{Definition}
\newtheorem{lma}{Lemma}
\newtheorem{assu}{Assumptions}
\newtheorem*{assum}{Assumptions}
\newtheorem{prop}{Proposition}
\newtheorem{cro}{Corollary}
\newtheorem{example}{Example}
\newcommand{\exm}{\begin{example}}
\newcommand{\exmm}{\end{example}}
\newtheorem*{theorem*}{Theorem}
\newcommand{\cor}{\begin{cro}}
\newcommand{\corr}{\end{cro}}
\newtheorem{exa}{Example}
\newcommand{\ex}{\begin{exa}}
\newcommand{\exx}{\end{exa}}
\newtheorem{remak}{Remark}
\newcommand{\rmk}{\begin{remak}}
\newcommand{\rmkk}{\end{remak}}
\newcommand{\thm}{\begin{tm}}
\newcommand{\nt}{\noindent}
\newcommand{\thmm}{\end{tm}}
\newcommand{\lm}{\begin{lma}}
\newcommand{\lmm}{\end{lma}}
\newcommand{\ass}{\begin{assu}}
\newcommand{\asss}{\end{assu}}
\newcommand{\assm}{\begin{assum}}
\newcommand{\assmm}{\end{assum}}
\newcommand{\df}{\begin{dfn}  }
\newcommand{\dff}{\end{dfn}}
\newcommand{\prp}{\begin{prop}}
\newcommand{\prpp}{\end{prop}}
\newcommand{\bqu}{\sloppy \small \begin{quote}}
\newcommand{\equ}{\end{quote} \sloppy \large}
\newcommand\cites[1]{\citeauthor{#1}'s\ (\citeyear{#1})}

\newcommand{\eq}{\begin{equation}}
\newcommand{\eqq}{\end{equation}}
\newtheorem{claim}{\it Claim}
\newcommand{\cl}{\begin{claim}}
\newcommand{\cll}{\end{claim}}
\newcommand{\bit}{\begin{itemize}}
\newcommand{\eit}{\end{itemize}}
\newcommand{\ben}{\begin{enumerate}}
\newcommand{\een}{\end{enumerate}}
\newcommand{\bcen}{\begin{center}}
\newcommand{\ecen}{\end{center}}
\newcommand{\fn}{\footnote}
\newcommand{\ds}{\begin{description}}
\newcommand{\dss}{\end{description}}
\newcommand{\prf}{\begin{proof}}
\newcommand{\prff}{\end{proof}}
\newcommand{\cs}{\begin{cases}}
\newcommand{\css}{\end{cases}}
\newcommand{\ml}{\item}
\newcommand{\lb}{\label}
\newcommand{\ra}{\rightarrow}
\newcommand{\tra}{\twoheadrightarrow}
\newcommand{\tlra}{\relbar\joinrel\twoheadrightarrow}
\newcommand*{\supp}{\operatornamewithlimits{sup}\limits}
\newcommand*{\inff}{\operatornamewithlimits{inf}\limits}
\newcommand{\nf}{\normalfont}
\renewcommand{\Re}{\mathbb{R}}
\newcommand{\Ze}{\mathbb Z}
\newcommand*{\mmax}{\operatornamewithlimits{max}\limits}
\newcommand*{\mmin}{\operatornamewithlimits{min}\limits}
\newcommand*{\argmax}{\operatornamewithlimits{arg max}\limits}
\newcommand*{\argmin}{\operatornamewithlimits{arg min}\limits}

\newcommand{\CR}{\mathcal R}
\newcommand{\CC}{\mathcal C}
\newcommand{\CT}{\mathcal T}
\newcommand{\GG}{{\cal G}}


\newtheorem{innercustomthm}{Theorem}
\newenvironment{customthm}[1]
  {\renewcommand\theinnercustomthm{#1}\innercustomthm}
  {\endinnercustomthm}
\newtheorem{einnercustomthm}{Extended Theorem}
\newenvironment{ecustomthm}[1]
  {\renewcommand\theeinnercustomthm{#1}\einnercustomthm}
  {\endeinnercustomthm}

\newcommand{\red}{\textcolor{red}}
\newcommand{\blue}{\textcolor{blue}}
\newcommand{\purple}{\textcolor{purple}}
\newcommand{\mred}[1]{\color{red}{#1}\color{black}}
\newcommand{\mblue}[1]{\color{blue}{#1}\color{black}}
\newcommand{\mpurple}[1]{\color{purple}{#1}\color{black}}
\makeatletter
\newcommand{\customlabel}[2]{%
\protected@write \@auxout {}{\string \newlabel {#1}{{#2}{}}}}
\makeatother
%

\def\qed{\hfill\vrule height4pt width4pt
depth0pt}
\def\reff #1\par{\noindent\hangindent =\parindent
\hangafter =1 #1\par}
\def\title #1{\begin{center}
{\Large {\bf #1}}
\end{center}}
\def\author #1{\begin{center} {\large #1}
\end{center}}
\def\date #1{\centerline {\large #1}}
\def\place #1{\begin{center}{\large #1}
\end{center}}

\def\date #1{\centerline {\large #1}}
\def\place #1{\begin{center}{\large #1}\end{center}}
\def\intr #1{\stackrel {\circ}{#1}}
\def\R{{\rm I\kern-1.7pt R}}
 \def\N{{\rm I}\hskip-.13em{\rm N}}
 \newcommand{\cprod}{\Pi_{i=1}^\ell}
\let\Large=\large
\let\large=\normalsize

%==============================================
%==============================================

\begin{titlepage}
\def\thefootnote{\fnsymbol{footnote}}
\vspace*{1.1in}

\title{DSGE- OSE Bootcamp 2019}

 %\vskip 2.50em

\author{Aniruddha Ghosh\fn{Department of Economics, Johns Hopkins University, Baltimore, MD 21218}}

\vskip 1.00em

\date{\today}

\vskip 2.75em


\baselineskip=.18in

\noindent Contains answers to the exercises on DSGE (Part-2 $\&$ 3), taught by Kerk Phillips as a part of the OSE Lab Summer Camp 2019. 




\end{titlepage}

%==============================================


\pagebreak

\setstretch{1.3}
\setcounter{footnote}{0}
\setcounter{secnumdepth}{5}
%==============================================


%%%%----------------------------------------------------------------------------%%%%
\section{Homework 2}
\subsection*{Exercise 3 (Dynamic Behavior)}

We do the necessary matrix algebra here to transform
\begin{equation}\label{uhlig}
E_t [F \tilde{X}_{t+1} + G  \tilde{X}_t + H  \tilde{X}_{t-1} + L  \tilde{Z}_{t+1} + M  \tilde{Z}_t]  = 0
\end{equation}
into 

\begin{equation}\label{uhlig}
[(FP + G)P + H] \tilde{X}_{t-1} + [(FQ + L)N + (FP + G)Q + M]\tilde{Z}_t
\end{equation}

By observation, we can see that eq.(2) is expressed in $\tilde{X}_{t-1}$ and $\tilde{Z}_{t}$. We substitute in eq.(1) the expression for $\tilde{X}_{t+1}$ and $\tilde{Z}_{t+1}$.

\begin{equation}
E_t [F \tilde{X}_{t+1} + G  \tilde{X}_t + H  \tilde{X}_{t-1} + L  \tilde{Z}_{t+1} + M  \tilde{Z}_t]=0  \\
\end{equation}

\begin{equation}
E_t [ F (P( P  \tilde{X}_{t-1} + Q  \tilde{Z}_t) + Q( N  \tilde{Z}_{t} + \epsilon_{t+1})) + G ( P  \tilde{X}_{t-1} + Q  \tilde{Z}_t) + H  \tilde{X}_{t-1} + L(N  \tilde{Z}_{t-1} + \epsilon_t) + M  \tilde{Z}_t \\]=0
\end{equation}

\begin{equation}
[(FP + G)P + H] \tilde{X}_{t-1} + [(FQ + L)N + (FP + G)Q + M]\tilde{Z}_t=0
\end{equation}

\section{Homework 3}
\subsection*{Exercise 1 (Perturbation Methods)}

We differentiate expression (5) involving terms with \textit{F(x,u)} with respect to \textit{u} and find,
\begin{equation}
x_{uuu} = -\frac{F_{uuu}+F_{xxx}x^3_u + 3(F_{xxu}x^2_u + F_{uux}x_u + F_{xu}x_{uu} + F_{xx}x_ux_{uu}) }{F_x}
\end{equation}
where the derivative is evaluated at  $u_0$, $x(u_0)$.
\end{document} 