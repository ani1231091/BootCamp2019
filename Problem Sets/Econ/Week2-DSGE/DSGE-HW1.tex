\documentclass[12pt]{article}
%\topmargin -0.5in \textheight 600pt\textwidth 465pt \oddsidemargin
%10pt\evensidemargin 10pt
\topmargin -0.8in \textheight 650pt\textwidth 490pt \oddsidemargin
-8pt\evensidemargin 0pt

\usepackage{graphicx}
\usepackage{epstopdf}
\usepackage{amsthm}
\usepackage{amsmath}
\usepackage{amssymb}
\usepackage{amstext}
\usepackage{amsfonts}
\usepackage{enumerate}
\usepackage[authoryear,nonamebreak,bibstyle]{natbib}
%\usepackage{geometry, pdflscape}
\usepackage{footnote}
\makesavenoteenv{tabular}
\makesavenoteenv{table}
%\usepackage{rotating}
%\usepackage[onehalfspacing]{setspace}
\usepackage[shortlabels]{enumitem}
\usepackage{multirow}
\usepackage[title, toc, page]{appendix}
\usepackage{xcolor}
\usepackage{setspace}
\usepackage{cancel}
\usepackage{soul}

%\usepackage[toc,page]{appendix}

\usepackage{sectsty}
\subsubsectionfont{\normalfont\itshape}

\begin{document}
\baselineskip=.22in\parindent=30pt

\newtheorem{tm}{Theorem}
\newtheorem{dfn}{Definition}
\newtheorem{lma}{Lemma}
\newtheorem{assu}{Assumptions}
\newtheorem*{assum}{Assumptions}
\newtheorem{prop}{Proposition}
\newtheorem{cro}{Corollary}
\newtheorem{example}{Example}
\newcommand{\exm}{\begin{example}}
\newcommand{\exmm}{\end{example}}
\newtheorem*{theorem*}{Theorem}
\newcommand{\cor}{\begin{cro}}
\newcommand{\corr}{\end{cro}}
\newtheorem{exa}{Example}
\newcommand{\ex}{\begin{exa}}
\newcommand{\exx}{\end{exa}}
\newtheorem{remak}{Remark}
\newcommand{\rmk}{\begin{remak}}
\newcommand{\rmkk}{\end{remak}}
\newcommand{\thm}{\begin{tm}}
\newcommand{\nt}{\noindent}
\newcommand{\thmm}{\end{tm}}
\newcommand{\lm}{\begin{lma}}
\newcommand{\lmm}{\end{lma}}
\newcommand{\ass}{\begin{assu}}
\newcommand{\asss}{\end{assu}}
\newcommand{\assm}{\begin{assum}}
\newcommand{\assmm}{\end{assum}}
\newcommand{\df}{\begin{dfn}  }
\newcommand{\dff}{\end{dfn}}
\newcommand{\prp}{\begin{prop}}
\newcommand{\prpp}{\end{prop}}
\newcommand{\bqu}{\sloppy \small \begin{quote}}
\newcommand{\equ}{\end{quote} \sloppy \large}
\newcommand\cites[1]{\citeauthor{#1}'s\ (\citeyear{#1})}

\newcommand{\eq}{\begin{equation}}
\newcommand{\eqq}{\end{equation}}
\newtheorem{claim}{\it Claim}
\newcommand{\cl}{\begin{claim}}
\newcommand{\cll}{\end{claim}}
\newcommand{\bit}{\begin{itemize}}
\newcommand{\eit}{\end{itemize}}
\newcommand{\ben}{\begin{enumerate}}
\newcommand{\een}{\end{enumerate}}
\newcommand{\bcen}{\begin{center}}
\newcommand{\ecen}{\end{center}}
\newcommand{\fn}{\footnote}
\newcommand{\ds}{\begin{description}}
\newcommand{\dss}{\end{description}}
\newcommand{\prf}{\begin{proof}}
\newcommand{\prff}{\end{proof}}
\newcommand{\cs}{\begin{cases}}
\newcommand{\css}{\end{cases}}
\newcommand{\ml}{\item}
\newcommand{\lb}{\label}
\newcommand{\ra}{\rightarrow}
\newcommand{\tra}{\twoheadrightarrow}
\newcommand{\tlra}{\relbar\joinrel\twoheadrightarrow}
\newcommand*{\supp}{\operatornamewithlimits{sup}\limits}
\newcommand*{\inff}{\operatornamewithlimits{inf}\limits}
\newcommand{\nf}{\normalfont}
\renewcommand{\Re}{\mathbb{R}}
\newcommand{\Ze}{\mathbb Z}
\newcommand*{\mmax}{\operatornamewithlimits{max}\limits}
\newcommand*{\mmin}{\operatornamewithlimits{min}\limits}
\newcommand*{\argmax}{\operatornamewithlimits{arg max}\limits}
\newcommand*{\argmin}{\operatornamewithlimits{arg min}\limits}

\newcommand{\CR}{\mathcal R}
\newcommand{\CC}{\mathcal C}
\newcommand{\CT}{\mathcal T}
\newcommand{\GG}{{\cal G}}


\newtheorem{innercustomthm}{Theorem}
\newenvironment{customthm}[1]
  {\renewcommand\theinnercustomthm{#1}\innercustomthm}
  {\endinnercustomthm}
\newtheorem{einnercustomthm}{Extended Theorem}
\newenvironment{ecustomthm}[1]
  {\renewcommand\theeinnercustomthm{#1}\einnercustomthm}
  {\endeinnercustomthm}

\newcommand{\red}{\textcolor{red}}
\newcommand{\blue}{\textcolor{blue}}
\newcommand{\purple}{\textcolor{purple}}
\newcommand{\mred}[1]{\color{red}{#1}\color{black}}
\newcommand{\mblue}[1]{\color{blue}{#1}\color{black}}
\newcommand{\mpurple}[1]{\color{purple}{#1}\color{black}}
\makeatletter
\newcommand{\customlabel}[2]{%
\protected@write \@auxout {}{\string \newlabel {#1}{{#2}{}}}}
\makeatother
%

\def\qed{\hfill\vrule height4pt width4pt
depth0pt}
\def\reff #1\par{\noindent\hangindent =\parindent
\hangafter =1 #1\par}
\def\title #1{\begin{center}
{\Large {\bf #1}}
\end{center}}
\def\author #1{\begin{center} {\large #1}
\end{center}}
\def\date #1{\centerline {\large #1}}
\def\place #1{\begin{center}{\large #1}
\end{center}}

\def\date #1{\centerline {\large #1}}
\def\place #1{\begin{center}{\large #1}\end{center}}
\def\intr #1{\stackrel {\circ}{#1}}
\def\R{{\rm I\kern-1.7pt R}}
 \def\N{{\rm I}\hskip-.13em{\rm N}}
 \newcommand{\cprod}{\Pi_{i=1}^\ell}
\let\Large=\large
\let\large=\normalsize

%==============================================
%==============================================

\begin{titlepage}
\def\thefootnote{\fnsymbol{footnote}}
\vspace*{1.1in}

\title{DSGE- OSE Bootcamp 2019}

 %\vskip 2.50em

\author{Aniruddha Ghosh\fn{Department of Economics, Johns Hopkins University, Baltimore, MD 21218}}

\vskip 1.00em

\date{\today}

\vskip 2.75em


\baselineskip=.18in

\noindent Contains answers to the exercises on DSGE (Part-1), taught by Kerk Phillips as a part of the OSE Lab Summer Camp 2019. 




\end{titlepage}

%==============================================


\pagebreak

\setstretch{1.3}
\setcounter{footnote}{0}
\setcounter{secnumdepth}{5}
%==============================================


%%%%----------------------------------------------------------------------------%%%%
\subsection*{Exercise 1}
The guess for the policy function in this Brock and Mirman setting takes the following form: $K_{t+1} = A e^{z_t} K_t^{\alpha}$. This is a very standard guess and is the conventional form of solution in Brock and Mirman model. The Euler equation is written as (expressed in notes),

\begin{equation}
\frac{1}{e^{z_t}K_t^{\alpha} - K_{t+1}} = \beta E_t \left[ \frac{\alpha e^{z_{t +1}} K_{t+1}^{\alpha -1}}{ e^{z_{t +1}} K_{t+1}^{\alpha} - K_{t+2}}  \right]
\end{equation}
Through method of undetermined coefficients, we can plug $K_{t+1} = A e^{z_t} K_t^{\alpha}$ into LHS and RHS and express \textit{A} in the form of model parameters. 
\begin{equation}
 \frac{1}{e^{z_t}K_t^{\alpha} - A e^{z_t} K_t^{\alpha}} = \frac{1}{e^{z_t} K_t^{\alpha} (1 - A)}
\end{equation}

For the RHS could be expressed as,
\begin{align}
\beta E_t [ \frac{\alpha e^{z_{t +1}} (A e^{z_t} K_t^{\alpha})^{\alpha -1}}{ e^{z_{t +1}} (A e^{z_t} K_t^{\alpha})^{\alpha} - Ae^{z_{t+1}} (Ae^{z_t} K_t^{\alpha})^{\alpha}}]\
&= \frac{\alpha \beta}{Ae^{z_t}K_t^{\alpha}(1-A)}
\end{align}

Comparing both sides gives us $A = \alpha \beta$. Therefore, the policy function can be expressed as  $k_{t+1} = H(k_t, z_t) = \alpha \beta e^{z_t} k_t^{\alpha}$.
This is corroborated from Stokey-Lucas and Sargent (??).

\subsection*{Exercise 2}

The contrast here is the fact that now we have intra-temporal substitution as well apart from intertemporal substitution. We have to optimize on two margins (consumption vs leisure, today vs tomorrow)  and therefore, the previous treatment will not be applicable in this context.

We have,
\begin{align}
u(c_t, l_t) &= \ln c_t+ a \ln(1 - l_t) && \text{King and Plosser for BGP }\\
F(K_t, L_t, z_t) &= e^{z_t} K_t^{\alpha}L_t^{1-\alpha} && \text{Stochastic production Function}
\end{align}\\

\vskip 3cm
The equations are as follows:
%Then, the seven equations characterizing equations and seven unknowns: $\{c_t, k_t, l_t, w_t, r_t, T_t, z_t\}$ for the model are as follows:

\begin{align}
c_t &= (1 - \tau)[w_tl_t + (r_t -\delta)k_t] + k_t + T_t - k_{t+1} \\
\frac{1}{c_t} &= \beta E_t\left[\frac{1}{c_{t+1}}[(r_{t+1} - \delta)(1 - \tau) + 1]\right] \\
\frac{a}{1-l_t} &= \frac{1}{c_t} w_t (1-\tau) \\
r_t &= \alpha e^{z_t} k_t^{\alpha -1} l_t^{1-\alpha} =  \alpha e^{z_t}  \left(\frac{l_t}{k_t} \right)^{1 - \alpha} \\
w_t &= (1 - \alpha) e^{z_t} k_t^{\alpha} l_t ^{-\alpha} = (1 - \alpha) e^{z_t} \left(\frac{k_t}{l_t} \right)^{\alpha} \\	
T_t &= \tau[w_tl_t + (r_t - \delta)k_t] \\
z_t &= (1-\rho_z)\bar{z} + \rho_z z_{t-1} + \epsilon^z_t 
\end{align}

Typical, seven variables and seven equation system.

\subsection*{Exercise 3}

The contrast from the previous setup is that now we admit for precautionary saving/risk aversion for the consumer. Earlier parts could be thought as a special case for this setting with $\gamma=1$. Now, we are in a setting which allows risk aversion to play a role. Also, SE and YE effects could be different depending on $\gamma$ therefore, governing response of consumption to productivity shocks. 

We have,

\begin{align}
u(c_t, l_t) &= \frac{c_t^{1-\gamma}-1}{1 - \gamma} + a \ln(1 - l_t) && \text{Separable and allows for Risk aversion } \\
F_(K_t, L_t, z_t) &= e^{z_t} K_t^{\alpha}L_t^{1-\alpha} && \text{Same as Before }
\end{align}


\begin{align}
c_t &= (1 - \tau)[w_tl_t + (r_t -\delta)k_t] + k_t + T_t - k_{t+1} \\
c_t ^{-\gamma} &= \beta E_t\left[ c_{t+1}^{-\gamma} [(r_{t+1} - \delta)(1 - \tau) + 1]\right] \\
\frac{a}{1-l_t} &= c_t ^{-\gamma} w_t (1-\tau) \\
r_t &= \alpha e^{z_t} k_t^{\alpha -1} l_t^{1-\alpha} =  \alpha e^{z_t}  \left(\frac{l_t}{k_t} \right)^{1 - \alpha} \\
w_t &= (1 - \alpha) e^{z_t} k_t^{\alpha} l_t ^{-\alpha} = (1 - \alpha) e^{z_t} \left(\frac{k_t}{l_t} \right)^{\alpha} \\	
T_t &= \tau[w_tl_t + (r_t - \delta)k_t] \\
z_t &= (1-\rho_z)\bar{z} + \rho_z z_{t-1} + \epsilon^z_t
\end{align}

\subsection*{Exercise 4}
I am liking the variation that is being offered here. This problem incorporates elasticity of labor (micro estimates small), one can easily see that the Frish elasticity of labor is $\frac{1}{\xi}$. Moreover, in this case we have the Dixit-Stiglitz production function with $\eta$ being the elasticity of substitution between capital and labor.

We have,
\begin{align*}
u(c_t, l_t) &= \frac{c_t^{1-\gamma} - 1}{1- \gamma} + a \frac{(1 - l_t)^{1 - \xi}-1}{1 - \xi} \\
F(K_t, L_t, z_t) &= e^{z_t}[\alpha K^{\eta}_t + (1 - \alpha)L^{\eta}_t]^{\frac{1}{\eta	}}
\end{align*}


\begin{align}
c_t &= (1 - \tau)[w_tl_t + (r_t -\delta)k_t] + k_t + T_t - k_{t+1} \\
c_t ^{-\gamma} &= \beta E_t\left[ c_{t+1}^{-\gamma} [(r_{t+1} - \delta)(1 - \tau) + 1]\right] \\
\frac{a}{(1-l_t)^{\xi}} &= c_t ^{-\gamma} w_t (1-\tau) \\
r_t &= \alpha e^{z_t} k_t^{\eta - 1} [\alpha k_t^{\eta} + (1 - \alpha) l_t^{\eta}]^{\frac{1 - \eta}{\eta}} \\
w_t &= (1 - \alpha) e^{z_t} l_t^{\eta - 1} [\alpha k_t^{\eta} + (1 - \alpha) l_t^{\eta}]^{\frac{1 - \eta}{\eta}} \\
T_t &= \tau[w_tl_t + (r_t - \delta)k_t] \\
z_t &= (1-\rho_z)\bar{z} + \rho_z z_{t-1} + \epsilon^z_t
\end{align}

\subsection*{Exercise 5}
Given,
\begin{align*}
u(c_t) &= \frac{c_t^{1-\gamma} - 1}{1- \gamma} \\
F(K_t, L_t, z_t) &= K_t^{\alpha} (L_t e_t^{z_t})^{1 - \alpha}
\end{align*}
 By the labor market clearing condition, we have that $L_t = l_t = 1$. The equations characterising the model are:
\begin{align}
c_t &= (1 - \tau)[w_t + (r_t - \delta)k_t] + k_t + T_t - k_{t+1} \\
c_t ^{-\gamma} &= \beta E_t \left[ c_{t+1}^{-\gamma} [(r_{t+1} - \delta)(1 - \tau) + 1] \right] \\
r_t &= \alpha k_t ^{\alpha - 1} (l_t e_t^{z_t})^{1 - \alpha} = \alpha \left(\frac{e^{z_t}}{k_t} \right)^{1 - \alpha} \\
w_t &= (1 - \alpha) k_t ^{\alpha} (e_t^{z_t})^{1 - \alpha} \\
T_t &= \tau [w_t + (r_t - \delta)k_t ] \\
z_t &= (1 - \rho_z)\bar{z} + \rho_z z_{t-1} + \epsilon^z_t  
\end{align}

Given the equations above, we can derive the steady state forms as below:
\begin{align}
\bar{c} &= (1 - \tau)[\bar{w}+ (\bar{r} - \delta) \bar{k}]  + \bar{T}  \\
\bar{T} &= \tau [\bar{w} + (\bar{r}- \delta)\bar{k} ] \\
\bar{c} ^{-\gamma} &= \beta E_t \left[ \bar{c}^{-\gamma} [(\bar{r} - \delta)(1 - \tau) + 1] \right] \\
\bar{r} &= \alpha \bar{k} ^{\alpha - 1} (e^{\bar{z}})^{1 - \alpha} = \alpha \left(\frac{e^{\bar{z}}}{\bar{k}} \right)^{1 - \alpha} \\
\bar{w} &= (1 - \alpha) \bar{k} ^{\alpha} (e^{\bar{z}})^{1 - \alpha} \\
\bar{z} &= (1 - \rho_z)\bar{z} + \rho_z \bar{z} + \epsilon^z_t
\end{align}

Deriving their steady state values is easy and they can be expressed out as,
\begin{align}
\bar{r} &= \frac{1 - \beta}{\beta (1 - \tau)} + \delta \\
\bar{k} &= \left(  \frac{\bar{r}}{\alpha}\right)^{\frac{1}{\alpha - 1}} \\
\bar{c} &= (1 - \tau)[\bar{w}+ (\bar{r} - \delta) \bar{k}]  \bar{T}  \\
\bar{w} &= (1 - \alpha) \bar{k} ^{\alpha} \\
\bar{T} &= \tau [\bar{w} + (\bar{r}- \delta)\bar{k} ] \\
\end{align}

\subsection*{Exercise 6}
Given,

\begin{align*}
u(c_t, l_t) &= \frac{c_t^{1-\gamma} - 1}{1- \gamma} + a \frac{(1 - l_t)^{1 - \xi}-1}{1 - \xi} \\
F(K_t, L_t, z_t) &= K_t^{\alpha} (L_t e^{z_t})^{1 - \alpha}
\end{align*}
As shown previously above, we have,
\begin{align}
c_t &= (1 - \tau)[w_tl_t + (r_t -\delta)k_t] + k_t + T_t - k_{t+1} \\
c_t ^{-\gamma} &= \beta E_t\left[ c_{t+1}^{-\gamma} [(r_{t+1} - \delta)(1 - \tau) + 1]\right] \\
w_t &= (1 - \alpha) e^{z_t} \left( \frac{k_t}{l_t e^{z_t}} \right)^{\alpha} \\
T_t &= \tau[w_tl_t + (r_t - \delta)k_t] \\
\frac{a}{(1-l_t)^{\xi}} &= c_t ^{-\gamma} w_t (1-\tau) \\
r_t &= \alpha \left( \frac{l_t e^{z_t}}{k_t} \right)^{1 - \alpha} \\
z_t &= (1-\rho_z)\bar{z} + \rho_z z_{t-1} + \epsilon^z_t
\end{align}

The steady state version of these equations are:
\begin{align}
\bar{c} &= (1 - \tau)[\bar{w} \bar{l} + (\bar{r} -\delta)\bar{k}] + \bar{T}  \\
\ 1 &= \beta E_t\left[ [(\bar{r} - \delta)(1 - \tau) + 1]\right] \\
\frac{a}{(1-\bar{l})^{\xi}} &= \bar{c} \bar{w} (1-\tau) \\
\bar{r} &= \alpha \left( \frac{\bar{l} e^{\bar{z}}}{\bar{k}} \right)^{1 - \alpha} \\
\bar{w} &= (1 - \alpha) e^{\bar{z}} \left( \frac{\bar{k}}{\bar{l} e^{\bar{z}}} \right)^{\alpha} \\
\bar{T} &= \tau[\bar{w} \bar{l} + (\bar{r} - \delta)\bar{k}] \\
\bar{z} &= (1-\rho_z)\bar{z} + \rho_z \bar{z} + \epsilon^{\bar{z}}
\end{align}


\end{document} 